\documentclass[letterpaper,10pt]{article}
\usepackage[utf8]{inputenc}
\usepackage{amsmath, amssymb}
\usepackage{centernot}
\usepackage{tikz}

\newcommand{\NaN}{\text{NaN}}

\title{Affinely extended reals with signed zero and NaN}
\author{Bill Zorn}

\begin{document}

\maketitle

\begin{abstract}
 We present a theory for the ideal arithmetic that underlies IEEE 754 floating point. This theory is based on ``typical'' real arithmetic over $\mathbb{R}$; it is affinely extended to include $-\infty$ and $\infty$; it represents a recoverable sign for all numbers, including 0; and it adds a special member NaN to represent values that are not a real number in the typical sense. It serves as a bridge between real arithmetic in a mathematical sense and specifications and implementaions of floating point formats, particularly IEEE 754. The theory can also specify ideal, infinite-precision results of floating point computations.
\end{abstract}

\begin{center}
 \huge \color{red} Work In Progress:

\large This vessel is not yet seaworthy.
\normalsize \normalcolor
\end{center}

\section{Notation}

 We denote the set of mathematical real numbers as $\mathbb{R}$. The affinely extended real numbers with signed zero and NaN, the set $\mathbb{R} \cup \{-\infty, \infty, -0, \NaN\}$, we denote $\mathbb{R}^*$.
 
 Numbers in $\mathbb{R}$ we write as simple variables, $a, b, x, y$, while numbers in $\mathbb{R}^*$ we distinguish with an overline, $\overline{a}, \overline{b}, \overline{x}, \overline{y}$. A written $a$ cannot be -0 or $\infty$, while a written $\overline{x}$ can.
 
 We distinguish arithmetic operations in the same way we distinguish numbers. $a + b$ denotes typical addition of real numbers, while $\overline{x} \;\overline{+}\; \overline{y}$ denotes addition under the arithmetic of extended real numbers with signed zero and \NaN, which we define in this document. $a \;\overline{+}\; y$ is well-defined (and hopefully not too different from typical addition), but $\overline{x} + \overline{y}$ is not, as the operands might not be real numbers.
 
\section{Ordering}
 
 $\mathbb{R}^*$ is only partially ordered. For all numbers in $\mathbb{R}^*$ that are also in $\mathbb{R}$, the typical rules for comparison of real numbers apply. The following rules define the ordering of the special members $-\infty$, $\infty$, and $-0$:
\begin{align}
 -\infty &< \overline{x} \in \mathbb{R}^* \setminus \{-\infty, \NaN\} \\
 \overline{x} \in \mathbb{R}^* \setminus \{\infty, \NaN\} &< \infty \\
 -0 &= 0 \\
 a < 0 \iff a &< -0 \\
 0 < a \iff -0 &< a
\end{align}
 Equality of members in $\mathbb{R}^* \setminus \{\NaN\}$ is reflexive, symmetric, and transitive, as is to be expected, and the comparisons $<$ and $>$ follow the usual symmetries. $-\infty = -\infty$, $a \in \mathbb{R} > 0 \iff a > -0$, and so on.

 \NaN is unordered with respect to all other elements of $\mathbb{R}^*$. The following rules define comparisons with it:
\begin{align}
 \NaN &\not< \overline{x} \\
 \NaN &\not> \overline{x} \\
 \NaN &\not= \overline{x}
\end{align}
 Note that equality is not reflexive ($\NaN \not= \NaN$), and that the usual symmetry of $<$ and $>$ is broken ($\overline{x} \not< \NaN \centernot\iff \overline{x} > \NaN$.

 Because $\mathbb{R}^*$ is only partially ordered, it is useful to define another notion of identicallity, $\overline{x} \equiv \overline{y}$, which captures when two members of $\mathbb{R}^*$ are actually the same element. $\overline{x} \equiv \overline{x}$ with no exceptions, even when $\overline{x} \equiv \NaN$, and $0 \not\equiv -0$ even though $0 = -0$.

\section{Primitives: negation, absolute value, and sign}

 All numbers in $\mathbb{R}^*$ have a recoverable sign, except for NaN, which can be thought of as not a number. Unary negation flips this sign, as follows:
\begin{align}
 neg(-\infty) &\equiv \infty \\
 neg(\infty) &\equiv -\infty \\
 neg(-0) &\equiv 0 \\
 neg(0) &\equiv -0 \\
 neg(a \in \mathbb{R} \setminus \{0\}) &\equiv -a \\
 neg(\NaN) &\equiv \NaN
\end{align}
 Absolute value forces this sign to be positive:
\begin{align}
 abs(-\infty) &\equiv \infty \\
 abs(\infty) &\equiv \infty \\
 abs(-0) &\equiv 0 \\
 abs(a) &\equiv |a| \\
 abs(\NaN) &\equiv \NaN
\end{align}
 We also define a sign operation to recover the sign, multiplied by 1:
\begin{align}
 sign(-\infty) &= -1 \\
 sign(\infty) &= 1 \\
 sign(-0) &= -1 \\
 sign(a) &=
 \begin{cases}
  -1, &a < 0 \\
  1,  &a = 0 \lor 0 < a \\
 \end{cases}\\
 sign(\NaN) &= -1 \;[]\; 1
\end{align}
 $-1 \;[]\; 1$ denotes a nondeterministic choice between -1 and 1. The sign of NaN is not undefinied: it must be -1 or 1, though this theory does not specify which one of those it is.
 
 Finally, we can transfer the sign from one number to another:
\begin{align}
 copysign(\overline{x}, \overline{y}) \equiv \overline{z} \iff (abs(\overline{x}) \equiv abs(\overline{z}) \land sign(\overline{y}) = sign(\overline{z})) \label{eq:copysign}
\end{align}
This definition runs into some trouble with NaN. To be faithful to the IEEE 754 standard, Equation \ref{eq:copysign} must always be satisfied, even for NaN, so the nondeterministic choice must always be made in a coherent way. This could be accomplished simply by choosing a single sign for NaN (say $sign(NaN) = 1$), or by tracking each NaN in a computation and assigning some specific sign to it.

\section{Addition and subtraction}

 All arithmetic operations in $\mathbb{R}^*$ are closed, with a defined result in $\mathbb{R}^*$. Addition is symmetric:
\begin{align}
 \overline{x} \;\overline{+}\; \overline{y} \equiv \overline{y} \;\overline{+}\; \overline{x}
\end{align}
 We take advantage of this to simplify the rules somewhat:
\begin{align}
 \NaN \;\overline{+}\; \overline{x} &\equiv \NaN \\
 \infty \;\overline{+}\; -\infty &\equiv \NaN \\
 \infty \;\overline{+}\; \overline{x} \in \mathbb{R}^* \setminus \{-\infty, \NaN\} &\equiv \infty \\
 -\infty \;\overline{+}\; \overline{x} \in \mathbb{R}^* \setminus \{\infty, \NaN\} &\equiv -\infty \\
 -0 \;\overline{+}\; \overline{x} \in \mathbb{R}^* \setminus \{0\} &\equiv \overline{x} \\
 -0 \;\overline{+}\; 0 &\equiv
 \begin{cases}
  -0, &\text{roundTowardNegative} \\
  0,  &\lnot \text{roundTowardNegative} \\
 \end{cases}\\
 a \;\overline{+}\; b &\equiv
 \begin{cases}
  a + b, &a + b \not= 0 \\
  -0,    &a + b = 0 \land \text{roundTowardNegative} \\
  0,     &a + b = 0 \land \lnot \text{roundTowardNegative} \\
 \end{cases}
\end{align}
 roundTowardNegative is some arbitrary predicate representing the IEEE 754 rounding mode. If it is True, then addition producing 0 should produce -0 instead, in accordance with the IEEE specification for arithmetic with that rounding mode. It can be thought of as a parameter of this theory, or as a distinction between two similar theories.

 Subtraction is defined in terms of addition and negation:
\begin{align}
 \overline{x} \;\overline{-}\; \overline{y} \equiv \overline{x} \;\overline{+}\; neg(\overline{y})
\end{align}

\section{Multiplication}

 Like addition, multiplication over $\mathbb{R}^*$ is symmetric:
\begin{align}
 \overline{x} \;\overline{\times}\; \overline{y} \equiv \overline{y} \;\overline{\times}\; \overline{x}
\end{align}
 It is defined as follows:
\begin{align}
 \NaN \;\overline{\times}\; \overline{x} &\equiv \NaN \\
 \overline{x} \in \{-0, 0\} \;\overline{\times}\; \overline{y} \in \{-\infty, \infty\} &\equiv \NaN \\
 \overline{x} \in \{-\infty, \infty\} \;\overline{\times}\; \overline{y} \in \mathbb{R}^* \setminus \{-0, 0, NaN\} &\equiv copysign(\infty, sign(\overline{x}) \times sign(\overline{y})) \label{eq:m1} \\
 \overline{x} \in \{-0, 0\} \;\overline{\times}\; \overline{y} \in \mathbb{R}^* \setminus \{-\infty, \infty, NaN\} &\equiv copysign(0, sign(\overline{x}) \times sign(\overline{y})) \label{eq:m2} \\
 a \;\overline{\times}\; b &\equiv a \times b
\end{align}
 Where it produces an answer other than NaN, multiplication effectively computes the sign and magnitude of the result separately; hence the use of $copysign$. This dependence could be removed by rewriting Equations \ref{eq:m1} and \ref{eq:m2} as cases on the signs of the operands, though this would be much more verbose.

\section{Division}

 Division is similar to multiplication, but it cannot be defined in terms of it. We cannot simply state some equality like $\overline{x} \;\overline{\div}\; \overline{y} \equiv \overline{x} \;\overline{\times}\; reciprocol(\overline{y})$. Instead, we define it as follows:
\begin{align}
 \NaN \;\overline{\div}\; \overline{x} &\equiv \NaN \\
 \overline{x} \;\overline{\div}\; \NaN &\equiv \NaN \\
 \overline{x} \in \{-\infty, \infty\} \;\overline{\div}\; \overline{y} \in \{-\infty, \infty\} &\equiv \NaN \\
 \overline{x} \in \{-0, 0\} \;\overline{\div}\; \overline{y} \in \{-0, 0\} &\equiv \NaN \\
 \overline{x} \in \{-\infty, \infty\} \;\overline{\div}\; \overline{y} \in \mathbb{R}^* \setminus \{-\infty, \infty, NaN\} &\equiv copysign(\infty, sign(\overline{x}) \times sign(\overline{y})) \label{eq:d1} \\
 \overline{x} \in \mathbb{R}^* \setminus \{-\infty, \infty, NaN\} \;\overline{\div}\; \overline{y} \in \{-\infty, \infty\} &\equiv copysign(0, sign(\overline{x}) \times sign(\overline{y})) \\
 \overline{x} \in \{-0, 0\} \;\overline{\div}\; \overline{y} \in \mathbb{R}^* \setminus \{-0, 0, NaN\} &\equiv copysign(0, sign(\overline{x}) \times sign(\overline{y})) \\
 \overline{x} \in \mathbb{R}^* \setminus \{-0, 0, NaN\} \;\overline{\div}\; \overline{y} \in \{-0, 0\} &\equiv copysign(\infty, sign(\overline{x}) \times sign(\overline{y})) \label{eq:d4} \\
 a \;\overline{\div}\; b &\equiv a \div b
\end{align}
 Unsurprisingly, division over $\mathbb{R}^*$ is not symmetric. Independent sign and magnitude computations occur, as with multiplication, in Equations \ref{eq:d1} - \ref{eq:d4}.
 
 Division can be used to compute something like the reciprocol, but there isn't a closed reciprocol in $\mathbb{R}^*$: $1 \;\overline{\div}\; \infty \equiv 0$, but $\infty \;\overline{\times}\; 0 \equiv \NaN$. In fact, it's impossible to find any $\overline{x} \in \mathbb{R}^*$ such that $\infty \;\overline{\times}\; \overline{x} \equiv 1$ or $0 \;\overline{\times}\; \overline{x} \equiv 1$.

\section{Integer powers}

 Integer powers are convenient for computing the real values of floating point numbers. The implementation here is equivalent to iterated multiplication (and division, in the case of negative exponents) for all exponents except 0. Note that the second argument is an integer, not a real number or a number in $\mathbb{R}^*$. We define the integer power function $pown$ as follows:

\begin{align}
 pown(\overline{x}, 0) &\equiv 1 \\
 pown(\NaN, n \in \mathbb{Z} \setminus \{0\}) &\equiv \NaN \\
 pown(\overline{x} \in \{\infty, -\infty\}, n \in \mathbb{Z} \setminus \{0\}) &\equiv
 \begin{cases}
  0,                              &\text{$n$ negative and even} \\
  copysign(0, \overline{x}),      &\text{$n$ negative and odd} \\
  \infty,                         &\text{$n$ positive and even} \\
  copysign(\infty, \overline{x}), &\text{$n$ positive and odd}
 \end{cases} \\
 pown(\overline{x} \in \{0, -0\}, n \in \mathbb{Z} \setminus \{0\}) &\equiv
 \begin{cases}
  \infty,                         &\text{$n$ negative and even}\\
  copysign(\infty, \overline{x}), &\text{$n$ negative and odd}\\
  0,                              &\text{$n$ positive and even}\\
  copysign(0, \overline{x}),      &\text{$n$ positive and odd}
 \end{cases} \\
 pown(x, n \in \mathbb{Z} \setminus \{0\}) &\equiv x^n
\end{align}

\section{Integer roots}

An integer root function is useful to complete the mandated ``single-ulp'' correctly rounded operations in the IEEE 754 standard, + - * / and sqrt. This has not yet been implemented for Titanic, as it is not needed to convert between floating point and real numbers.

More details coming soon!

\section{Other operations}

The operations defined as of this document's compilation on \today\; are sufficient to implement the required correctly rounded arithmetic operations of the IEEE 754 standard: addition, subtraction, multiplication, division, squareRoot, and fusedMultiplyAdd. The defined integer powers and roots also implement the corresponding IEEE 754 recommended operations.

Future versions of this document will add other definitions, with the intention of covering all IEEE 754 recommended operations and the common mathematical operations present in the FPCore standard.

%\bibliographystyle{ieeetr}
%\bibliography{ref}

\end{document}